\documentclass[presentation]{beamer}

\usepackage[utf8]{inputenc}
% \usepackage[T1]{fontenc}
\usepackage{fixltx2e}
\usepackage{graphicx}
% \usepackage{longtable}
% \usepackage{tabu}
\usepackage{makecell}
\usepackage{float}
\usepackage{subcaption}
\usepackage{wrapfig}
\usepackage{rotating}
\usepackage[normalem]{ulem}
\usepackage{amsmath}
% \usepackage{textcomp}
% \usepackage{marvosym}
% \usepackage{wasysym}
% \usepackage{amssymb}
\usepackage{hyperref}
\usepackage{ragged2e}
\usepackage{xcolor}


\usepackage{bm}

\usepackage{amsmath,amssymb}
\newcommand{\bx}{\mathbf{x}}
\newcommand{\bv}{\mathbf{v}}
\newcommand{\bp}{\mathbf{p}}
\newcommand{\bq}{\mathbf{q}}
\newcommand{\by}{\mathbf{y}}

\newcommand{\bbR}{\mathbb{R}}
\newcommand{\bbN}{\mathbb{N}}

\newcommand{\bxi}{\bm{\xi}}
\newcommand{\bal}{\bm{\alpha}}
\newcommand{\bth}{\bm{\theta}}

\usepackage{etoolbox}
\let\bbordermatrix\bordermatrix
\patchcmd{\bbordermatrix}{8.75}{4.75}{}{}
\patchcmd{\bbordermatrix}{\left(}{\left[}{}{}
\patchcmd{\bbordermatrix}{\right)}{\right]}{}{}

\usepackage[authoryear, round]{natbib}

\newcommand{\sidebysidecaption}[4]{%
\RaggedRight%
  \begin{minipage}[t]{#1}
    \vspace*{0pt}
    #3
  \end{minipage}
  \hfill%
  \begin{minipage}[t]{#2}
    \vspace*{0pt}
    #4
\end{minipage}%
}

%% colors
\definecolor{Red}{rgb}{0.7,0,0}
\definecolor{Blue}{rgb}{0,0,0.8}
\definecolor{Green}{rgb}{0,0.8,0}
\definecolor{Gray}{rgb}{0.8,0.8,0.8}
\definecolor{Orange}{rgb}{1,.65,0}

\usepackage{hyperref}
\hypersetup{
  hyperindex = {true},
  colorlinks = {true},
  linktocpage = {true},
  plainpages = {false},
  linkcolor = {Blue},
  citecolor = {Blue},
  urlcolor = {Red},
  pdfstartview = {Fit},
  pdfpagemode = {UseOutlines},
  pdfview = {XYZ null null null},
  pdfkeywords={proteomics, spatial, dynamics, integration, regulation, computational biology},
  pdfsubject={Research talk},
  pdfcreator={Laurent Gatto}}

%% knitr

%% maxwidth is the original width if it is less than linewidth
%% otherwise use linewidth (to make sure the graphics do not exceed the margin)
\makeatletter
\def\maxwidth{ %
  \ifdim\Gin@nat@width>\linewidth
    \linewidth
  \else
    \Gin@nat@width
  \fi
}
\makeatother

\definecolor{fgcolor}{rgb}{0.345, 0.345, 0.345}
\newcommand{\hlnum}[1]{\textcolor[rgb]{0.686,0.059,0.569}{#1}}%
\newcommand{\hlstr}[1]{\textcolor[rgb]{0.192,0.494,0.8}{#1}}%
\newcommand{\hlcom}[1]{\textcolor[rgb]{0.678,0.584,0.686}{\textit{#1}}}%
\newcommand{\hlopt}[1]{\textcolor[rgb]{0,0,0}{#1}}%
\newcommand{\hlstd}[1]{\textcolor[rgb]{0.345,0.345,0.345}{#1}}%
\newcommand{\hlkwa}[1]{\textcolor[rgb]{0.161,0.373,0.58}{\textbf{#1}}}%
\newcommand{\hlkwb}[1]{\textcolor[rgb]{0.69,0.353,0.396}{#1}}%
\newcommand{\hlkwc}[1]{\textcolor[rgb]{0.333,0.667,0.333}{#1}}%
\newcommand{\hlkwd}[1]{\textcolor[rgb]{0.737,0.353,0.396}{\textbf{#1}}}%

\usepackage{framed}
\makeatletter
\newenvironment{kframe}{%
 \def\at@end@of@kframe{}%
 \ifinner\ifhmode%
  \def\at@end@of@kframe{\end{minipage}}%
  \begin{minipage}{\columnwidth}%
 \fi\fi%
 \def\FrameCommand##1{\hskip\@totalleftmargin \hskip-\fboxsep
 \colorbox{shadecolor}{##1}\hskip-\fboxsep
     % There is no \\@totalrightmargin, so:
     \hskip-\linewidth \hskip-\@totalleftmargin \hskip\columnwidth}%
 \MakeFramed {\advance\hsize-\width
   \@totalleftmargin\z@ \linewidth\hsize
   \@setminipage}}%
 {\par\unskip\endMakeFramed%
 \at@end@of@kframe}
\makeatother

\definecolor{shadecolor}{rgb}{.97, .97, .97}
\definecolor{messagecolor}{rgb}{0, 0, 0}
\definecolor{warningcolor}{rgb}{1, 0, 1}

\definecolor{errorcolor}{rgb}{1, 0, 0}
\newenvironment{knitrout}{}{} % an empty environment to be redefined in TeX

\usepackage{alltt}
\usepackage[utf8]{inputenc}
\usepackage[T1]{fontenc}
\usepackage{fixltx2e}
\usepackage{graphicx}
\usepackage{longtable}
\usepackage{float}
\usepackage{wrapfig}
\usepackage{rotating}
\usepackage[normalem]{ulem}
\usepackage{amsmath}
\usepackage{textcomp}
\usepackage{marvosym}
\usepackage{wasysym}
\usepackage{amssymb}


\date{13 June 2019 -- The Alan Turing Institute \\ \textit{Workshop on
    Statistical Data Science for\\ Proteomics and Metabolomics}}

\title{Probabilistic modelling of \\ protein sub-cellular localisation}

\author{Laurent Gatto\\
  \url{laurent.gatto@uclouvain.be}\\
  \url{http://lgatto.github.io/about}\\
  de Duve Institute -- UCLouvain\\
  \vspace{2mm}
  Slides: \url{http://bit.ly/20190613Turing}
}


% \AtBeginSection[] % Do nothing for \section*
% {
%   \begin{frame}<beamer>
%     \frametitle{Plan}
%     \tableofcontents[currentsection]
%   \end{frame}
% }


\begin{document}

\maketitle


\include{spatprot}


\section{Computational infrastructure}

\begin{frame}{}
  \begin{center}
    \Large{\textbf{Behind the scenes}: software/data structures and
      open research practice.}
  \end{center}
\end{frame}


\begin{frame}{}

  Beyond the figures\footnote{... which are all reproducible, by the way.}

  \begin{itemize}
  \item Software: \textbf{infrastructure}
    (\href{http://bioconductor.org/packages/MSnbase}{\texttt{MSnbase}},
    \cite{Gatto:2012}), \textbf{dedicated machine learning}
    (\href{http://bioconductor.org/packages/pRoloc}{\texttt{pRoloc}},
    \cite{Gatto:2014a}), \textbf{interactive
      visualisation}\footnote{\url{https://lgatto.shinyapps.io/christoforou2015/}}
    (\href{http://bioconductor.org/packages/pRolocGUI}{\texttt{pRolocGUI}},
    \cite{pRolocGUI}) and \textbf{data}
    (\href{http://bioconductor.org/packages/pRolocdata}{\texttt{pRolocdata}},
    \cite{Gatto:2014a}) for spatial proteomics.
  \item \textbf{Open source}, and \textbf{coordinated and
    collaborative development}, enabling \textbf{reproducible
    research}, enables understanding of the data (not a black box) and
    \textbf{drive scientific innovation}.
  \end{itemize}
\end{frame}


\begin{frame}[fragile]{Conclusions}
  \begin{itemize}
  \item Protein sub-cellular localisation: technologies (hyperLOPIT)
    and opportunities.

  \item Reliance on computational biology, statistics and dedicated
    software (\texttt{pRoloc} \textit{et al.}) to interpret data and
    acquire biological knowledge.

  \item Rigorous computational infrastructure and sound data analysis
    and interpretation is a \textbf{long term investment}.

  \end{itemize}

\end{frame}


\begin{frame}[allowframebreaks]{References}
  \scriptsize
  \bibliographystyle{plainnat}
  \bibliography{refs}
\end{frame}


\begin{frame}
  \begin{block}{Acknowledgements}
    \begin{itemize}
    \item \textbf{Mr Oliver Crook} and Dr Lisa Breckels, (U
      of Cambridge): spatial proteomics, machine learning, software.
    \item Dr Sebastian Gibb and Dr Johannes Rainer: MS and proteomics
      software.
    \item Prof Kathryn Lilley (U of Cambridge), Dr Claire Mulvey,
      (CRUK Cambridge Institute): data.
    \item Funding: BBSRC, Wellcome Trust.
    \end{itemize}

    Slides at \url{http://bit.ly/20190613Turing} under CC-BY SA.
 
  \end{block}

  
  \begin{center}
    \textbf{Thank you for your attention}
  \end{center}

\end{frame}

\end{document}
